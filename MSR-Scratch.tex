\documentclass[10pt, conference]{IEEEtran}

% correct bad hyphenation here
%\hyphenation{op-tical net-works semi-conduc-tor}

\usepackage{graphicx}
\usepackage{dcolumn}
\usepackage{color}

\newcommand{\fenia}[1]{\emph{\color{blue}Fenia says: #1}}
\newcommand{\felienne}[1]{\emph{\color{green}Felienne says: #1}}
\newcommand{\grex}[1]{\emph{\color{yellow}Gregorio says: #1}}
\newcommand{\jemole}[1]{\emph{\color{red}Jesús says: #1}}

\begin{document}

\title{A dataset of Scratch programs,\\Scraped, Shaped and Graded}


\author{
	\IEEEauthorblockN{Efthimia Aivaloglou\IEEEauthorrefmark{1}, Felienne Hermans\IEEEauthorrefmark{1}, Gregorio Robles\IEEEauthorrefmark{2}, and Jes{\'u}s Moreno-Le{\'o}n\IEEEauthorrefmark{3}}
	
	\IEEEauthorblockA{\IEEEauthorrefmark{1}Delft University of Technology, The Netherlands\\
			Email: \{e.aivaloglou, f.f.j.hermans\}@tudelft.nl}
	\IEEEauthorblockA{\IEEEauthorrefmark{2}Universidad Rey Juan Carlos, Fuenlabrada (Madrid), Spain \\
		Email: grex@gsyc.urjc.es}
	\IEEEauthorblockA{\IEEEauthorrefmark{3}Programamos.es \& Universidad Rey Juan Carlos, Spain\\
		Email: jesus.moreno@programamos.es}
}

\maketitle


\begin{abstract}
In this paper, we present a collection of 250K Scratch programs scraped from the Scratch project repository. We processed the program data to encode it into a database that facilitates querying and further analysis. The dataset is intended as an input for research on computing education and source code analysis.
\end{abstract}

\begin{IEEEkeywords}
Scratch; dataset;
\end{IEEEkeywords}



 
\section{Introduction}
\cite{Aivaloglou_2016}
\cite{Robles_2017}

\section{Dataset Construction}
a description of the methodology used to gather it (preferably with the tool used to create/generate the data)

\section{Data Representation}
a description of the data source
a description of the storage mechanism, including a schema if applicable,

\section{Using the Dataset / Enabled Research / Research Opportunities}
a description of how the data has been used by others,

\section{Extending the Dataset}
ideas for what future research questions could be answered or what further improvements could be made to the data set

\section{Limitations}
any limitations and/or challenges in creating or using this data set.

\section{Conclusion}

\bibliographystyle{IEEEtran}
\bibliography{IEEEabrv,ScratchDataset}

\end{document}


