\documentclass[10pt, conference]{IEEEtran}

% correct bad hyphenation here
%\hyphenation{op-tical net-works semi-conduc-tor}

\usepackage{url}
\usepackage{graphicx}
\usepackage{dcolumn}
\usepackage{color}

\newcommand{\nPrograms}{250,163}
\newcommand{\nAnalyzedPrograms}{247,798}
\newcommand{\nemptyPrograms}{14,307}
\newcommand{\nScriptPrograms}{233,491}
\newcommand{\nLOC}{36,085,654}
\newcommand{\nscripts}{4,049,356}


\newcommand{\fenia}[1]{\emph{\color{blue}Fenia says: #1}}
\newcommand{\felienne}[1]{\emph{\color{green}Felienne says: #1}}
\newcommand{\grex}[1]{\emph{\color{yellow}Gregorio says: #1}}
\newcommand{\jemole}[1]{\emph{\color{red}Jesús says: #1}}

\begin{document}

\title{A dataset of Scratch programs,\\Scraped, Shaped and Scored}


\author{
	\IEEEauthorblockN{Efthimia Aivaloglou\IEEEauthorrefmark{1}, Felienne Hermans\IEEEauthorrefmark{1}, Gregorio Robles\IEEEauthorrefmark{2}, and Jes{\'u}s Moreno-Le{\'o}n\IEEEauthorrefmark{3}}
	
	\IEEEauthorblockA{\IEEEauthorrefmark{1}Delft University of Technology, The Netherlands\\
			Email: \{e.aivaloglou, f.f.j.hermans\}@tudelft.nl}
	\IEEEauthorblockA{\IEEEauthorrefmark{2}Universidad Rey Juan Carlos, Fuenlabrada (Madrid), Spain \\
		Email: grex@gsyc.urjc.es}
	\IEEEauthorblockA{\IEEEauthorrefmark{3}Programamos.es \& Universidad Rey Juan Carlos, Spain\\
		Email: jesus.moreno@programamos.es}
}

\maketitle


\begin{abstract}
In this paper, we present a collection of 250K Scratch programs scraped from the Scratch project repository. We processed the program data to encode it into a database that facilitates querying and further analysis. The dataset is intended as an input for research on computing education and source code analysis.
\end{abstract}

\begin{IEEEkeywords}
Scratch; dataset;
\end{IEEEkeywords}


 
\section{Introduction}
Scratch \cite{resnick_scratch:_2009} is a block-based programming language developed to serve as a stepping stone for children from 8 to 16 years old to the more advanced world of computer programming.
It offers a web-based programing environment that enables creating games and interactive animations. The public repository of Scratch programs contains over 19 million projects and 16 million users.

A number of works in the computing education research field attempt to assess the programming skills that novice programmers develop in the Scratch environment.
Some utilize program data collected during specific programming courses (e.g., \cite{meerbaum-salant_learning_2010, wilson_evaluation_2012, Maloney_2008}), while others utilize the dataset made available by the Lifelong Kindergarten Group at the MIT Media Lab, which contains data for Scratch projects created until 2012 (e.g., \cite{fields_2014, yang_2015, Dasgupta_2016}).
In addition to identifying indications of learning of programming concepts, static analysis of Scratch programs has also been performed for identifying code smells and bad programming practices within Scratch programs \cite{Meerbaum_habits_2011, Aivaloglou_2016}, and automated quality assessment tools have been proposed (e.g. Hairball \cite{boe_hairball:_2013} and Dr. Scratch \cite{moreno_automatic_2014}).

While Scratch is receiving increasing interest as an introductory programming language, there is no recent dataset of Scratch programs available to the research community.
The one made available from the MIT Media Lab concerns projects created using the previous, initial version of the Scratch application, before the introduction of the Scratch 2 web programming interface in 2013.
It is since then that the popularity of Scratch started increasing\footnote{Monthly activity trends can be found at \url{https://scratch.mit.edu/statistics/}}.

The goal of this paper is to present an open and timely dataset of recent Scratch programs, along with their metadata, that can facilitate quantitative research in the fields of source code analysis and computing education.
The dataset contains 250 thousand Scratch projects, from 100 thousand different users, that were scraped from the Scratch project repository.
It is made available as a database\footnote{{add anonymized address}} which includes, for each Scratch project, its metadata and the program data, along with programming mastery scoring results from the Dr. Scratch quality assessment tool \cite{moreno_automatic_2014}.

\section{Dataset Construction}
\label{dataset}

\subsection{Data collection}
To collect the data from the web interface of the Scratch project repository we built a scraping program.
The web scraping program, called Kragle, starts by reading the Scratch projects page\footnote{\label{scratchpublic}\url{https://scratch.mit.edu/explore/projects/all/}} and thus obtains the project identifiers of projects that were most recently shared.
Subsequently, Kragle retrieves the JSON code for each of the listed projects\footnote{For a given project id $x$, the program's JSON representation can be obtained via \url{https://cdn.projects.scratch.mit.edu/internalapi/project/x/get}}.

We ran Kragle on March 2nd 2016 for 24 hours and, during that time, it obtained the JSON code for \nPrograms projects. Out of those, we failed to parse and further analyze 2,367 projects due to technical difficulties with the JSON files.
Kragle, as well as all scraped projects and our analysis files are available.\footnote{\label{repo}\url{https://github.com/TUDelftScratchLab/ScratchDataset}}

Once we obtained the Scratch projects, we parsed the JSON files according to the specification of the format\footnote{\url{http://wiki.scratch.mit.edu/wiki/Scratch_File_Format_(2.0)}}.
This resulted in a list of used blocks per project, within the sprites and the stage of the project.
We also cross referenced all blocks with the Scratch wiki to determine the shapes and the category of all blocks.
For example, \texttt{When Green Flag Clicked} is a \emph{Hat block} from the \emph{Events category}.

\subsection{Calculation of programming mastery scores}
\fenia{Description of how the mastery metrics for       
	[Abstraction]
	,[Parallelism]
	,[Logic]
	,[Synchronization]
	,[FlowControl]
	,[UserInteractivity]
	,[DataRepresentation]
	,[Mastery]
	,[Clones]
	,[CustomBlocks]
	,[InstancesSprites]
	were calculated}

\subsection{Importing the data}
\label{dataAnalysis}
All scraped project data and metadata, including the list of used blocks and parameters, were imported in a relational database.
We also imported the data on the shapes and the categories of the Scratch blocks.
We then used SQL queries for normalizing the data and bringing it in its final schema, which is drawn in Figure???

\section{Data Representation}
a description of the data source
a description of the storage mechanism, including a schema if applicable,

\begin{table}[ht]
	\centering
	\begin{tabular}{llp{5.4cm}}
		\textbf{Table}& \textbf{Key} & \textbf{Attribute(Description)}\\
		\hline
		\textbf{\texttt{project}} & PK & \texttt{p\_id} (Scratch project ID)\\
		&  & 							\texttt{project-name} (name given to project)\\
		&  & 							\texttt{username} (author's Scratch 
		username)\\
		&  & 							\texttt{total-views} (project views number)\\
		&  & 							\texttt{total-remixes} (project remixes number)\\
		&  & 							\texttt{total-favorites} (total users favoriting)\\
		&  & 							\texttt{total-loves} (total users `loving' the project)\\
		& & 							\texttt{is-remix} (calculated column, true if project is a remix of another one)\\
		\hline
		\textbf{\texttt{script}} & PK & \texttt{script\_id} (auto increment)\\
		& FK & 							\texttt{project\_id} (\scriptsize{\texttt{*\ldots1 project:p\_id}})\\
		& & 							\texttt{sprite-type} (one of [sprite, stage, procDef])\\
		& & 							\texttt{sprite-name} (name sprite the script is in)\\
		& & 							\texttt{script-rank} (project-level ranking of script)\\
		& & 							\texttt{coordinates} (x-y location of script in Scratch editor)\\
		& & 							\texttt{total-blocks} (calculated column, number of blocks comprising the script)\\
		\hline
		\textbf{\texttt{procedure}} & PK,FK & \texttt{script\_id} (\scriptsize{\texttt{1\ldots1 script:script\_id}})\\
		& & 							\texttt{proc-name} (name given to custom block)\\
		& & 							\texttt{total-args} (number of input arguments)\\
		\hline
		\textbf{\texttt{block}} & PK,FK  & \texttt{script\_id} (\scriptsize{\texttt{*\ldots1 script:script\_id}})\\
		& PK & \texttt{block-rank} (script-level ranking of block)\\
		& FK & \texttt{block-type} (\scriptsize{\texttt{*\ldots1 blockType:b-type}})\\
		&  & \texttt{parameter-1} (value of 1st input parameter)\\
		&  &\vdots\\
		&  & \texttt{parameter-24} (value of 24th input parameter)\\
		\hline
		\textbf{\texttt{blockType}} & PK & \texttt{b-type} (Scratch name of predefined block)\\
		&  & \texttt{category} (Scratch block category)\\
		&  & \texttt{shape} (One of the 8 block shapes in the Scratch interface)\\
		&  & \texttt{is-input} (true if the block receives user input)\\
		\hline
		\textbf{\texttt{remix}} & PK  & \texttt{remix\_id} (\texttt{\scriptsize{project\_id}} of created project)\\
		&  & \texttt{from-project\_id} (\texttt{\scriptsize{project\_id}} of original project)\\
		\hline
		\textbf{\texttt{grades}} & PK,FK & \texttt{project\_id} (\texttt{\scriptsize{1\ldots1 project:p\_id}})\\
		& & \texttt{Abstraction} ()\\
		& & \texttt{Parallelism} ()\\
		& & \texttt{Logic} ()\\
		& & \texttt{Synchronization} ()\\
		& & \texttt{FlowControl} ()\\
		& & \texttt{UserInteractivity} ()\\
		& & \texttt{DataRepresentation} ()\\
		& & \texttt{Mastery} ()\\
		& & \texttt{Clones} ()\\
		& & \texttt{CustomBlocks} ()\\
		& & \texttt{InstancesSprites} ()\\
		\hline
	\end{tabular}
	\caption{Database schema: Tables and Attributes}
	\label{tbl-dbschema}
\end{table}

\section{Using the Dataset / Enabled Research / Research Opportunities}
a description of how the data has been used by others,

\section{Extending the Dataset}
ideas for what future research questions could be answered or what further improvements could be made to the data set

\section{Limitations}
any limitations and/or challenges in creating or using this data set.

\section{Conclusion}

\bibliographystyle{IEEEtran}
\bibliography{IEEEabrv,ScratchDataset}

\end{document}


